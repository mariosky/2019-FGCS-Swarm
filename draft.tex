\documentclass[review]{elsarticle}

\usepackage{lineno,hyperref}

\modulolinenumbers[5]

\journal{Future Generation Computer Systems}
\begin{document}

\begin{frontmatter}

\title{An Event-based Architecture for Multi-population Optimization Algorithms}

\author[itt]{Mario Garci\'a Valdez \corref{mycorrespondingauthor}}
\cortext[mycorrespondingauthor]{Corresponding author}
\ead{mario@tectijuana.edu.mx}

\author[granada]{Juan J. Merelo Guerv\'os}
\ead{jmerelo@geneura.ugr.es}

\address[itt]{Department of Graduate Studies, Instituto Tecnol\'ogico de Tijuana, Tijuana BC, Mexico}
\address[granada]{Universidad de Granada, Granada, Spain}

\begin{abstract}
    Multi-population based methods have been used extensively in recent years to
    improve the performance of nature-inspired optimization algorithms.
    Researchers try to avoid premature convergence and maintain population
    diversity by dividing the original population into multiple small
    subpopulations. Each subpopulation then evolves independently using a
    specified search strategy. Finally, these subpopulations communicate with
    each other to exchange individuals. Multi-population based methods are
    parallel and asynchronous by nature; researchers take advantage of this to
    scale algorithms with distributed, multi-threaded, and parallel
    implementations. The design of parallel and distributed versions of
    multi-population algorithms for their execution in cloud infrastructures is
    not a trivial task. Designers need to adapt the algorithms to leverage the
    benefits of scalability, elasticity, fault-tolerance, reproducibility,  and
    cost-effectiveness of cloud-based systems. In this work, we present the
    design and implementation of a novel event-driven architecture, designed to
    distribute the processing of population-based algorithms asynchronously. The
    system provides researchers a serverless strategy for creating multiple
    populations with different parameters of execution, allowing the
    implementation of multiple algorithms. We provide a container-based platform
    where researchers can deploy experiments by just specifying the resources
    and parameters of the algorithm. The performance of the platform and the
    benefits of a multi-population approach has been evaluated using an ensemble
    of Genetic  Algorithms (GAs) and Particle Swarm Optimization (PSO).
    Experiments show that the framework allows the combined algorithms to
    outperform, with a high probability, single-algorithm versions. The
    framework we provide also has few parameters to tune since single-algorithm
    parameters are selected randomly. The architecture and the implemented
    platform is an excellent alternative for locally or a cloud-based
    deployment.
\end{abstract}

\begin{keyword}
Multi-population \sep Nature-inspired algorithm \sep Parallel Genetic Algorithms \sep Cloud-Computing
\sep Event-driven architecture 
\end{keyword}

\end{frontmatter}

\linenumbers

\section{Introduction}

\section{Bibliography styles}

There are various bibliography styles available. You can select the style of your choice in the preamble of this document. These styles are Elsevier styles based on standard styles like Harvard and Vancouver. Please use Bib\TeX\ to generate your bibliography and include DOIs whenever available.

Here are two sample references: \cite{Feynman1963118,Dirac1953888}.

\bibliographystyle{elsarticle-num}
\bibliography{mybibfile}

\end{document}