\documentclass[review]{elsarticle}

\usepackage{lineno,hyperref}

\modulolinenumbers[5]

\journal{Future Generation Computer Systems}
\begin{document}

\begin{frontmatter}

\title{An Event-based Architecture for Multi-population Optimization Algorithms}

\author[itt]{Mario Garci\'a Valdez \corref{mycorrespondingauthor}}
\cortext[mycorrespondingauthor]{Corresponding author}
\ead{mario@tectijuana.edu.mx}

\author[granada]{Juan J. Merelo Guerv\'os}
\ead{jmerelo@geneura.ugr.es}

\address[itt]{Department of Graduate Studies, Instituto Tecnol\'ogico de Tijuana, Tijuana BC, Mexico}
\address[granada]{Universidad de Granada, Granada, Spain}

\begin{abstract}
    Multi-population based optimization methods are parallel and asynchronous by
    nature; researchers take advantage of this to scale algorithms with
    distributed, multi-threaded, and parallel implementations. However, the design of
    parallel and distributed versions of multi-population algorithms for their
    execution in cloud infrastructures is not a trivial task. Designers need to
    adapt the algorithms to leverage the benefits of scalability, elasticity,
    fault-tolerance, reproducibility, and cost-effectiveness of reactive
    cloud-based systems. In this work, we present the design and
    implementation of a reactive container-based application for the execution
    of multi-population algorithms. The application can be used to execute
    multiple experiments, both locally or in a cloud environment. Researchers
    can deploy experiments by specifying the infrastructure as part of an
    experiment definition so that it is much easier to reproduce results. The
    performance of the platform and the benefits of a multi-population approach
    has been evaluated using an ensemble of Genetic  Algorithms (GAs) and
    Particle Swarm Optimization (PSO). Experiments show that the framework
    allows the combined algorithms to outperform, with a high probability,
    single-algorithm versions. The framework we provide also has fewer
    parameters to tune since single-algorithm parameters are selected randomly.
    The architecture and the implemented platform is an excellent alternative
    for locally or a cloud-based deployment.
\end{abstract}

\begin{keyword}
Multi-population \sep Nature-inspired algorithm \sep Parallel Genetic Algorithms \sep Cloud-Computing
\sep Event-driven architecture 
\end{keyword}

\end{frontmatter}

\linenumbers

\section{Introduction}

In the last decades, nature-inspired optimization algorithms have been succesfully
applied to solve complex real-world problems \cite{yang2014nature}. Algorithms
inspired in natural processes include evolutionary algorithms (EAs)
\cite{back1996evolutionary} and swarm intelligence (SI) \cite{kennedy2006swarm},
among others. These population-based algorithms share the common characteristic
of using an initial set of random candidate solutions that are later used to
generate a new set of candidates, using a nature-inspired heuristic. Popular EAs
are Genetic Algorithms (GAs), Genetic Programming (GP), grey wolf optimization
(GWO) and Differential Evolution (DE), while examples of (SI) are particle swarm
optimization (PSO) and ant colony algorithms (ACO).

As in nature, population-based algorithms are intrinsically parallel and
asynchronous. The fitness of each individual can be evaluated independently of
others, and similarly, each population could evolve in isolation. Since 
earlier works, researchers have been proposing some form of parallelization
\cite{muhlenbein1988evolution} to increase the scalability of these algorithms.
The island model was one of the first techniques proposed for parallelization,
which lead to an increased performance
\cite{gorges1990explicit,grosso1985computer}. The concept was to divide a large
population into communicated subpopulations. Since then, other population-based
algorithms have adopted the concept, and researchers have found other benefits
besides the execution speed; these include avoiding a premature convergence and
maintaining the diversity of the global population. Researchers use the term
multi-population based methods when generally referring to techniques using
subpopulations as part of their optimization strategy.

When designing efficient multi-population algorithms, researchers need to
consider additional issues. These include the number and size of subpopulations,
the interaction between them, the search area of subpopulations, and the search
strategy and parametrization of each subpopulation. We must take into account
these high-level issues when designing a parallel architecture. 

The parallel execution of multi-population algorithms has been the focus of many
works,  going from earlier hardware-based works using transputers,
multi-threaded, and multi-core systems. Recently the trend has been on
exploiting a higher number of processing units by using GPUs, or distributed
systems: client-server, web-based, map-reduce,  grid, and cloud computing. Cloud
computing is becoming the standard way of running enterprise applications, not
only because of the convenience of the pay-as-you-go model or the non-existent
cost of administration, but it also offers a way of describing the
infrastructure as part of the code, so that it is much easier to reproduce
results and has been a boon for scientific computing.  However,  cloud computing
has also been evolving, going from monolithic applications, that is,
applications built on a single stack of services that communicate by layers, to
microservice and serverless architectures that favor the parallel and
asynchronous communication of heterogeneous resources. In these architectures,
services or even functions are single processing points, that depart from the
monolithic or even distributed paradigm to become a loose collection of {\em
microservices} \cite{microservices}, which in many cases are stateless, that
react to events and they only exist while they are doing some processing. 

Based on these considerations, in this work, we present the design and
implementation of a reactive multi-population based platform extending the
architecture proposed in [Anonymous]; this is a so-called {\em serverless}
architecture that asynchronously processes isolated and heterogeneous
subpopulations. In the proposed reactive system, each subpopulation is trated as
a message that is pushed asynchronously into a message queue. Messages trigger
stateless functions that receive the subpopulation and proceed to run an
algorithm, using the parameters and population included in the message. After
the specified number of iterations, each stateless function returns the evolved
subpopulation by again pushing a message to another queue, used for receiving
the resulting subpopulations. Subpopulations are received from the queue by a
controller that is responsible for mixing the individuals from different
subpopulations and producing new subpopulations. These new subpopulations are
pushed again by the controller into the message queue, creating a loop. The
cycle stops when the controller receives a subpopulation containing a candidate
solution that satisfies a particular condition, or a maximum number of messages
were received. 

To evaluate the capability of a cross-breed multi-population solution,
we conducted several experiments using different benchmark functions, comparing the
results of single versus cross-breed multi-population algorithms. For the experiments, we choose to
compare the PSO and GA algorithms, as they are well understood, and there are
several implementations in the literature. We implemented both algorithms as
stateless functions, and more algorithms can be added in the same way.

The rest of the paper is organized as follows

\section{Multi-population design}

Multi-population based methods divide the original population into
smaller subpopulations or islands, with every subpopulation carrying out the
algorithm independently, with synchronous or asynchronous communication with the
rest of the islands. This relative isolation helps in maintaining an overall
diversity since each subpopulation will search in a particular area, at least
between communications. The recombination mentioned above (mixing) or migration
between subpopulations is needed to avoid a premature convergence of candidate
solutions since smaller populations are known to perform better for a given
problem than bigger populations. However, it gives them the added advantage of
parallel operation. Additionally, and in some cases, multi-population algorithms
scale better than expected due to the interaction between the algorithm and the
parallelism of the operation \cite{ALBA20027}.

However, in most cases, algorithms applied to each subpopulation are
homogeneous, or at any rate, the same variant of the algorithm. As long as this
parallel operation is not synchronous, other population-based algorithms, or, as
a matter of fact, any algorithm, could be easily integrated. That is why several
works based on multi-population are heterogeneous, integrating various
optimization algorithms, and often performing better than single-population or
homogeneous optimization algorithms \cite{wu2016differential,nseef2016adaptive}.

Heterogeneous algorithms add another degree of freedom to the problem of finding
the correct parameter settings for an algorithm; because some parameters affect
the accuracy of the solution and the convergence speed of the algorithms as they
tip the balance between exploration and exploitation of the search space. On the
other hand, current studies show that by having a high number of subpopulations
interacting in parallel, the effect of the individual parameters of each
subpopulation is compensated by those selected in other subpopulations. In this
work, we will use random settings within a specific range as results have shown
this is a valid solution to this problem. 

Some parameters, specially the population size, are
kept fixed in order to control more easily the execution of the algorithm. For
instance, by having the size of subpopulations fixed, it is easier to control
the number of evaluations and the communication costs, when the algorithm is in
operation.

Combining multiple algorithms, with different parameters, interacting with each
other at the same time, can benefit from the strengths of each. For instance, a
genetic algorithm could find a promising global solution that is not optimal
while another algorithm, more suitable for a local search, finds the global
optimum. This approach has been followed extensively in recent years, with
success. Moreover, there is a need for frameworks, architecture, and
implementation models that can allow researchers the development of new
parallel, asynchronous, heterogeneous, and parameter-free algorithms in a scalable way.  

\section{Bibliography styles}

There are various bibliography styles available. You can select the style of your choice in the preamble of this document. These styles are Elsevier styles based on standard styles like Harvard and Vancouver. Please use Bib\TeX\ to generate your bibliography and include DOIs whenever available.

Here are two sample references: \cite{Feynman1963118,Dirac1953888}.

\bibliographystyle{elsarticle-num}
\bibliography{mybibfile}

\end{document}